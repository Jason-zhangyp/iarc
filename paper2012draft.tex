\documentclass[12pt,letterpaper]{article}
\usepackage[margin=1in]{geometry}
\usepackage{fancyhdr}
\usepackage[utf8]{inputenc}
\usepackage{palatino}
\usepackage{microtype}
\usepackage{graphicx}
\usepackage{lastpage}
\usepackage[hang,small]{caption}

\renewcommand{\headrulewidth}{0pt}
\fancyfoot{}
\fancyfoot[C]{\sffamily Page \thepage\ of \pageref{LastPage}}
\pagestyle{fancy}

\usepackage{titlesec}
\titleformat{\section}{\Large\bfseries\MakeUppercase}{\thesection}{}{}
\titleformat{\subsection}{\large\bfseries}{\thesubsection}{}{}
\titleformat{\subsubsection}{\bfseries\itshape}{\thesubsubsection}{}{}

\setlength{\parindent}{0cm}
\setlength{\parskip}{1em}



\begin{document}

\begin{center}
{\huge
    OSURC Autonomous Aerial \\ Robotics Team\\[1cm]
    % TODO: need a more on-topic title.
}
{
Kyle Dillon \\
Daniel Miller \\
Tim Niedermeyer \\
Ryan Skeele \\
Michael Williamson \\
Soo-Hyun Yoo \\ \vspace{0.5em}

\emph{Oregon State University}
}

\end{center}

\section*{Abstract}

The Oregon State University Autonomous Aerial Robotics Team has developed
an indoor autonomous quadrotor with custom hardware and software to compete
in the International Aerial Robotics Competition (IARC). Onboard, an
ATXmega128a3 microcontroller runs a 200 Hz PD angular position controller
to maintain stable flight. The quadrotor is capable of sending live video,
LIDAR scans, and altitude measurements to an offboard Robot Operating
System (ROS) navigation system that processes the sensory information and
passes navigational commands back to the quadrotor. The quadrotor possesses
a passively compliant robotic hand capable of grasping objects of varying
shapes and sizes.


\section*{Introduction}

The Oregon State University strategy for indoor automous flight involves
specific functional goals for each part of the system. Flight stability is
handled by the flight platform, however, all navigational data gathered from
distance sensors and cameras mounted on the flight platform is transmitted to
the base station for processing. The base station then transmits navigational
commands to the flight platform.

Stabilization sensory is handled using a three axis gyroscope, 
accellerometer and magnetometer. Navigational data comes from a number of
distance sensors mounted on servos. This solution was chosen because of the
low electrical and computing power required to process distance sensor data.
A seperate wireless camera is also implemented to provide object recognition
through the OpenCV libraries.


\section*{Hardware}

\subsection*{Chassis}

The new chassis design will make use of lighter weight carbon fiber tubing.  The
 current design consists of 4 motors mounted on arms that mount to a central
chassis.  The new design will consist of only two solid arms that mount to one
another, each connecting a pair of motors.  This will remove bending moments
resulting from motor forces from the central chassis, allowing for material
reduction and weight savings.  The current chassis also has a tendency to snag
the ground when the aerial vehicle lands with significant lateral speed.  This
causes the vehicle to tip over, sending its propellers into the ground and
causing damage.  The new chassis will have less skid resistance, while allowing
for vertical impact absorption.  Blade guards will also be added around the
perimeter of the propellers, to reduce the risk of propeller damage from
incidental contact with the environment.


\subsection*{End Effector}

The IARC competition requires the use of a grasping mechanism for one of the
primary objectives.  The chosen design for the grasping mechanism will make use
of an under-actuated, passively compliant four-fingered hand.  Each of the four
fingers will contain two flexure joints, rather than rotational bearing joints.
A fully actuated hand would require eight actuators; this under-actuated design
will require only one.  A single actuator will control all joints through the
use of a pulley system, allowing each finger link to actuate until it comes into
 contact with an object.  All tendon cables will see the same force from the
actuator.  Not only does this design greatly reduce cost, but it saves
weight--an essential benefit for an aerial vehicle.  In addition, the design all
ows the hand to automatically adapt to the shape of any object, without the need
 for special positioning and calculated movements.


\subsection*{Electronics}

The electrical setup for our quad rotor runs primarily off of a 3 cell
lithium polymer battery. The ~12V from they battery is used to power the motors
and motor controllers on the flight platform. This supply is also regulated down
to 5V to power an Atmel Xmega microcontroller used as the central intelligence
on the flight platform, an xbee radio for wireless communication, and
addionally to power the various distance and inertial sensors. A wireless camera
is powered by a seperate 9V source.


\section*{Software}

\subsection*{Aerial stabilization and flight control}

% \section*{Flight Control}
% The on board flight control system was developed to provide basic
% stability for the flight platform. It does not have contingencies to prevent
% drift. Integration of gyroscope samples is used to create an estimate of the
% orientation of the flight platform. This combined with the raw data from the
% gyroscopes is as input to a position differential (PD) controller. To prevent
% the gyroscope positional estimates from drifting they are averaged with
% magnetometer and accelerometer samples.

\subsubsection*{Orientation kinematics}

An accurate measurement of the aerial vehicle's orientation is key to
autonomous stabilization. The orientation can be represented with a direction
cosine matrix (DCM), which is a 3x3 matrix containing the cosines between each
of the 9 possible pairs of axes of two separate Cartesian coordinate systems.

In the context of inertial measurement units in robotics applications, a 3D
vector could be represented in either the global (earth) or the body (local)
frames of reference. For example, the location of an end effector may be
represented as $<1, 0, 0>$ in the body frame but have different and changing
$x$, $y$, and $z$ components in the global frame, and vice versa.

If the controller loop frequency is high enough (on the order of 100 Hz), an
axis of the DCM, its rotational vector, and its linear velocity are
approximately orthogonal to each other. Thus, the magnitude of the angular
velocity of a unit vector approximately equals its linear velocity, which means
that the DCM can be calculated by integrating the gyroscope readings.

Unfortunately, since the gyroscope only measures changes in the orientation,
the DCM will drift over time. A 3-axis accelerometer can be used to correct the
roll and pitch drift, which will keep the vehicle upright.

In order to achieve this, we wish to keep the global Z vector codirectional
with the negative gravity vector. That is, the cross product of the two vectors
should equal zero, and we can use whatever the cross product actually is as our
correctional rotation vector. It is important to keep in mind, however, that
since the accelerometer reports acceleration in the body frame of reference, we
must express Z in the same frame of reference as the acceleration vector.

This body frame Z vector is simply found by taking the third row of the
transpose of the DCM. We cross multiply Z with the gravity vector, add the
resulting correction vector to the angular displacement vector, and take a
weighted average.

Small imperfections in the mounting of the accelerometer can be offset by a
rotation matrix.


\subsubsection*{PID control}

Some systems can be controlled with a simple P controller, where a given offset
is translated directly to some actuator output. Other systems, like the flight
orientation control of micro aerial vehicles (MAVs), may require a damping term
on the position to prevent a divergent oscillation of the vehicle's angular
position.


\subsubsection*{Cascading PID controllers}

The D controller in a simple angular position PD controller can hinder the
responsiveness of the MAV, which is crucial to stability. A better alternative
is an angular position P controller that feeds a desired velocity to an angular
velocity PD controller.

At first thought, it might seem that a position P/velocity PD controller is no
different than a position PD controller, since both the position D and the
velocity P are based on velocity measurements. However, the two controllers do
different things with the velocity measurements.

The position D controller has a damping effect on motion in that it always
resists a change in position. This means that if the MAV experiences a
perturbation away from some desired position, the D controller will help resist
the motion, as desired. However, as the MAV tries to recover from this error
state, the D controller blindly hinders the movement back towards the desired
position, which is not at all optimal.

The velocity P controller, on the other hand, pushes the velocity to whatever
it should be, whether that means slowing it down or speeding it up. The
velocity D controller ensures that the velocity change does not happen too
abruptly, helping reduce the chance of overshoot. The controller is able to do
this by keeping track of a desired velocity in addition to the current
velocity, which provides a context upon which the controller can ``decide''
whether it should help a positional movement or hinder it, instead of blindly
hindering all movement as is the case with a position D controller.

This means that the P/PD controller as a whole can take a desired position
input and accelerate the body towards and maintain a target angular velocity
until the current position nears the target position. Only then will the
controller actively slow down the movement. This makes for a controller that
can respond much more quickly while maintaining stability.

\end{document}
